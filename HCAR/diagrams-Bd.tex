% diagrams-Bd.tex
\begin{hcarentry}[updated]{diagrams}
\report{Brent Yorgey}%05/15
\status{active development}
\participants{many}
\makeheader

The diagrams framework provides an embedded domain-specific language
for declarative drawing.  The overall vision is for diagrams to become
a viable alternative to DSLs like MetaPost or Asymptote, but with the
advantages of being \emph{declarative}---describing what to draw, not
how to draw it---and \emph{embedded}---putting the entire power of
Haskell (and Hackage) at the service of diagram creation.  There is
still much more to be done, but diagrams is already quite
fully-featured, with a comprehensive user manual, a large collection
of primitive shapes and attributes, many different modes of
composition, paths, cubic splines, images, text, arbitrary monoidal
annotations, named subdiagrams, and more.

%**<img width=400 src="./arrows.jpg">
%*ignore
\begin{center}
\includegraphics[width=0.47\textwidth]{html/arrows.jpg}
\end{center}
%*endignore

\subsubsection*{What's new}

There has not yet been a new major release of diagrams since version
1.3 in April, but work has continued apace. Here is a sampling of new
features already in diagrams HEAD or currently being worked on:
\begin{compactitem}
\item B-spline support, and B-spline to cubic Bezier conversion
\item CSG support for 3D diagrams
\item New techniques and tools for drawing 2D projections of 3D diagrams
\item Constraint-based layout
\item Separate fill and stroke opacity attributes
\item New aligned composition operator
\end{compactitem}

\texttt{diagrams-pandoc}, a pandoc filter which can automatically
compile diagrams code included inline in pandoc documents, had its
first release to Hackage.

We are also working on using \texttt{stack} to create a system for
easier, more reproducible builds, which will benefit both users and
developers, and form the basis for much more comprehensive continuous
integration testing.

%**<img width=350 src="./kaleidoscope.jpg">
%*ignore
\begin{center}
\includegraphics[width=0.45\textwidth]{html/kaleidoscope.jpg}
\end{center}
%*endignore

\subsubsection*{Contributing}

There is plenty of exciting work to be done; new contributors are
welcome!  Diagrams has developed an encouraging, responsive, and fun
developer community, and makes for a great opportunity to learn and
hack on some ``real-world'' Haskell code.  Because of its size,
generality, and enthusiastic embrace of advanced type system features,
diagrams can be intimidating to would-be users and contributors;
however, we are actively working on new documentation and resources to
help combat this.  For more information on ways to contribute and how
to get started, see the Contributing page on the diagrams wiki:
\url{http://haskell.org/haskellwiki/Diagrams/Contributing}, or come
hang out in the \texttt{\#diagrams} IRC channel on freenode.

%**<img width=400 src="./topoform.jpg">
%*ignore
\begin{center}
\includegraphics[width=0.4\textwidth]{html/topoform.jpg}
\end{center}
%*endignore

\FurtherReading
\begin{compactitem}
\item \url{http://projects.haskell.org/diagrams}
\item \url{http://projects.haskell.org/diagrams/gallery.html}
\item \url{http://haskell.org/haskellwiki/Diagrams}
\item \url{http://github.com/diagrams}
\item \url{http://ozark.hendrix.edu/~yorgey/pub/monoid-pearl.pdf}
\item \url{http://www.youtube.com/watch?v=X-8NCkD2vOw}
\end{compactitem}
\end{hcarentry}
