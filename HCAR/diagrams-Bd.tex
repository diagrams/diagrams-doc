% diagrams-Bd.tex
\begin{hcarentry}[updated]{diagrams}
\report{Brent Yorgey}%05/11
\status{active development}
\participants{Ryan Yates}
\makeheader

The diagrams library provides an embedded domain-specific language for
declarative drawing.  The overall vision is for diagrams to become a
viable alternative to DSLs like MetaPost or Asymptote, but with the
advantages of being \emph{declarative}---describing what to draw, not
how to draw it---and \emph{embedded}---putting the entire power of
Haskell (and Hackage) at the service of diagram creation.

%**<img width=500 src="./paradox.jpg">
%*ignore
\begin{center}
\includegraphics[width=0.47\textwidth]{html/paradox.jpg}
\end{center}
%*endignore

Development on the library has proceeded apace since the last HCAR,
and the 0.4 release now features a comprehensive user manual as well
as support for a large collection of primitive shapes, many different
modes of composition, paths, cubic splines, images, text, arbitrary
monoidal annotations, named subdiagrams, and more.

There is plenty more work to be done; new contributors are
particularly welcome!

%**<img width=500 src="./triangular-numbers.jpg">
%*ignore
\begin{center}
\includegraphics[width=0.47\textwidth]{html/triangular-numbers.jpg}
\end{center}
%*endignore

\FuturePlans

Plans for the near future include a native SVG backend, improved font
support, arrowheads, and improvements to the handling of named
subdiagrams.  Longer-term plans include support for animations, a
custom Gtk application for editing diagrams, and any other awesome
stuff we think of.

\FurtherReading
\url{http://projects.haskell.org/diagrams}
\url{http://code.google.com/p/diagrams/issues/list}
\end{hcarentry}
