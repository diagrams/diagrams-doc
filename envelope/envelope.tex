\documentclass[11pt]{amsart}
\usepackage{geometry}                % See geometry.pdf to learn the layout options. There are lots.
\geometry{letterpaper}                   % ... or a4paper or a5paper or ... 
%\geometry{landscape}                % Activate for for rotated page geometry
%\usepackage[parfill]{parskip}    % Activate to begin paragraphs with an empty line rather than an indent
\usepackage{graphicx}
\usepackage{amssymb}
\usepackage{epstopdf}
\usepackage{amsthm}
\DeclareGraphicsRule{.tif}{png}{.png}{`convert #1 `dirname #1`/`basename #1 .tif`.png}
\setlength{\parindent}{0em}
\setlength{\parskip}{1em}
\newtheorem{defn}{Definition}
\newtheorem{thm}{Theorem}

\begin{document}
\begin{center}
\bigskip 
\textbf{Envelopes, Support Functions and the Convex Hull}

\textsc{Jeffrey Rosenbluth}

\textit{April 2015}
\end{center}
\hrule
\vspace{0.25in}

We define both envelopes and convex hulls in terms of \emph{support functions} and show
that the convex hull is a (possibly proper) subset of the envelope. We then make some observations
about envelopes. 

\section{Envelope vs Convex Hull}

We start with the definition of support functions
on  $\mathbb{R}^n$.
\footnote{
The definition of a support function can be generalized to arbitrary vector spaces.
Let $S$ be any set in a vector space $X$. We define the support function $q_S$ on the dual $X'$ of $X$ as 
$q_S(\ell) = sup_{s \in S} \ell(s)$
for any linear function $\ell.$ 
(see \emph{Linear Algebra}, by Peter Lax, 1997 or just about any  text book on linear functional analysis).
}
\begin{defn}[Support Function]
\label{supportfunc}
Let $S$ be a set in $\mathbb{R}^n$. We define the support function $q_S$ as 
$$q_S(v) = sup_{s \in S} \langle s,v \rangle$$
where $\langle \cdot, \cdot \rangle$ denotes inner product.
\end{defn}

Note that $q_S(v)$ may be $\infty$ for some $v$. 

In contrast to diagrams we will not distinguish between points and vectors, taking the definition of both to be elements in $\mathbb{R}^n$.
This allows us to fix the origin at $0$ and to consider two diagrams that are identical except for their origins to be translations of one another.

From the diagrams documentation and Definition \ref{supportfunc} we can rewrite the definition of an envelope 
in terms of its support function $q_d$.
\begin{defn}[Envelope]
The envelope of a diagram $d$ in direction $v$ is
\begin{align}
e_d(v) &= sup_{u \in d} \frac{\langle u, v \rangle}{\|v\|}\\
           &=sup_{u \in d} \langle u, v/\|v\| \rangle\\
           &=q_d(v/\|v\|)
\end{align}
\end{defn}

We can create an intensional representation of the envelope by taking the union over all $v$ of the sets
$$\{av : a \leq e_d(v)\} \cap \{ bv : b \leq e_d(-v)\}$$
Therefore if we can represent the convex hull as the union over all $v$ of the sets
$$\{av : a \leq c_d(v)\} \cap \{ bv : b \leq c_d(-v)\}$$
where $c_d(v) \leq e_d(v)$, then we will have shown the convex hull of $d$ is a subset of its envelope.

To define the convex hull in terms of the support function we use the following theorem. 
\footnote{\emph{Linear Algebra}, by Peter Lax, 1997, page 162}

\begin{thm}[Convex Hull]
\label{convexhull} The closed convex hull of any set $S$ is the set of points x satisfying $\langle u, x \rangle \leq q_S(u)$ for all $u$ in $\mathbb{R}^n$.
\end{thm}

\begin{thm}
The convex hull of a diagram is a subset of its envelope
\end{thm}
\textsc{Proof}: We consider all points in the convex hull of diagram $d$ that lie on some vector $v$, these points have the form $c_d(v)v$ for some
scalar $c_d(v)$. By Theorem \ref{convexhull}, for all $u$, these points must satisfy
\begin{align}
\langle u, c_d(v)v \rangle &\leq q_d(u)\\
c_d(v)\langle u, v \rangle &\leq q_d(u)\\
c_d(v) &\leq  \frac{q_d(u)}{\langle u, v \rangle}
\end{align}
Since the last inequality hold for all $u$ we have,
\begin{align}
c_d(v) &\leq \inf_u  \frac{q_d(u)}{\langle u, v \rangle}\\
&\leq \frac{q_d(v)}{\langle v, v \rangle} \\
&= q_d(v/\|v\|)= e_d(v) 
\end{align}
 and hence $c_d(v) \leq e_d(v)$ which is suffiicient to prove the Theorem as explained above. \qed
 
 Finally, in order to show that the convex hull may be a proper subset of the envelope it is enough to find some diagram $d$ and vector $v$ such that
 $c_d(v) < e_d(v)$. We take $d$ equal to the horizontal line from $(0,0)$ to $(1,0)$ in $\mathbb{R}^2$ and $v=(1,1)$. Then
 \begin{align}
 e_d((1,1)) &= sup_{x \in [0,1]} \frac{\langle (x,0), (1,1) \rangle}{\sqrt{2}}\\
                 &= sup_{x \in [0,1]} \frac{x}{\sqrt{2}}\\
                 &= \frac{1}{\sqrt{2}}
\end{align}
The convex hull of $d$ is $d$ since a line segment is convex, and the only point in $d$ lying on $(1,1)$ is $(0,0)$, hence $c_d((1,1)) = 0$ which is less than 
$1/\sqrt{2}$, thus $c_d((1,1)) < e_d((1,1))$.

\section{Observations}

Almost all diagrams have an envelope which is a proper subset of its convex hull, Figures 1,2 and 5.
In some cases the bounding box is a better approximation to the convex hull than the envelope, a square
for example.

\begin{figure}[h]
\label{s00}
 \centering
\includegraphics[width=150pt]{sq1_0_0.png}
\caption{Envelope of unit square centered at $(0,0)$}
\end{figure}

\begin{figure}[h]
\label{s05}
 \centering
\includegraphics[width=150pt]{sq1_05.png}
\caption{Envelope of unit square centered at $(0,-\frac{1}{2})$}
\end{figure}

The envelope of a diagram is highly dependent on its origin, Figures 3 and 4.
As the origin moves farther away from the center of the diagram, the envelope becomes a 
worse approximation to the convex hull.

\begin{figure}[h]
\label{s55}
 \centering
\includegraphics[width=150pt]{sq1_55.png}
\caption{Envelope of unit square centered at $(\frac{1}{2}, \frac{1}{2})$}
\end{figure}

\begin{figure}[h]
 \centering
\includegraphics[width=150pt]{sq1_11.png}
\caption{Envelope of unit square centered at $(5,5)$}
\end{figure}

\begin{figure}[h]
 \centering
\includegraphics[width=150pt]{c1_00.png}
\caption{Envelope of an ellipse}
\end{figure}

Like the convex hull the shape of the bounding box is independent of the placement of the origin. This is not true of envelopes.
Finally, I wonder if there is an efficient algorithm to calculate and extensional form of the convex hull using the variational form
$$c_d(v) = \inf_u  \frac{q_d(u)}{\langle u, v \rangle}$$

\end{document}  